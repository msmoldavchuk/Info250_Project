% Options for packages loaded elsewhere
\PassOptionsToPackage{unicode}{hyperref}
\PassOptionsToPackage{hyphens}{url}
%
\documentclass[
]{article}
\usepackage{amsmath,amssymb}
\usepackage{iftex}
\ifPDFTeX
  \usepackage[T1]{fontenc}
  \usepackage[utf8]{inputenc}
  \usepackage{textcomp} % provide euro and other symbols
\else % if luatex or xetex
  \usepackage{unicode-math} % this also loads fontspec
  \defaultfontfeatures{Scale=MatchLowercase}
  \defaultfontfeatures[\rmfamily]{Ligatures=TeX,Scale=1}
\fi
\usepackage{lmodern}
\ifPDFTeX\else
  % xetex/luatex font selection
\fi
% Use upquote if available, for straight quotes in verbatim environments
\IfFileExists{upquote.sty}{\usepackage{upquote}}{}
\IfFileExists{microtype.sty}{% use microtype if available
  \usepackage[]{microtype}
  \UseMicrotypeSet[protrusion]{basicmath} % disable protrusion for tt fonts
}{}
\makeatletter
\@ifundefined{KOMAClassName}{% if non-KOMA class
  \IfFileExists{parskip.sty}{%
    \usepackage{parskip}
  }{% else
    \setlength{\parindent}{0pt}
    \setlength{\parskip}{6pt plus 2pt minus 1pt}}
}{% if KOMA class
  \KOMAoptions{parskip=half}}
\makeatother
\usepackage{xcolor}
\usepackage[margin=1in]{geometry}
\usepackage{graphicx}
\makeatletter
\def\maxwidth{\ifdim\Gin@nat@width>\linewidth\linewidth\else\Gin@nat@width\fi}
\def\maxheight{\ifdim\Gin@nat@height>\textheight\textheight\else\Gin@nat@height\fi}
\makeatother
% Scale images if necessary, so that they will not overflow the page
% margins by default, and it is still possible to overwrite the defaults
% using explicit options in \includegraphics[width, height, ...]{}
\setkeys{Gin}{width=\maxwidth,height=\maxheight,keepaspectratio}
% Set default figure placement to htbp
\makeatletter
\def\fps@figure{htbp}
\makeatother
\setlength{\emergencystretch}{3em} % prevent overfull lines
\providecommand{\tightlist}{%
  \setlength{\itemsep}{0pt}\setlength{\parskip}{0pt}}
\setcounter{secnumdepth}{-\maxdimen} % remove section numbering
\ifLuaTeX
  \usepackage{selnolig}  % disable illegal ligatures
\fi
\usepackage{bookmark}
\IfFileExists{xurl.sty}{\usepackage{xurl}}{} % add URL line breaks if available
\urlstyle{same}
\hypersetup{
  pdftitle={An Anyalsis Of The Housing Market},
  hidelinks,
  pdfcreator={LaTeX via pandoc}}

\title{An Anyalsis Of The Housing Market}
\author{}
\date{\vspace{-2.5em}2025-08-17}

\begin{document}
\maketitle

The housing market in the U.S is very vital to understand. Those with
the finical ability to acquire a house are investing in an asset that
they perceive will one day grow in value. The money behind a housing
purchase is not small and with such a large purchase it is important to
understand the forces that drive the prices of that asset. This report
will detail historic price drivers in the housing market along with the
current force that is pushing prices. Index versions of graphs with a
base year of 2000 are included. These serve to highlight the
relationship between values however the values themselves are more
difficult to interpret. The focus is on the non indexed graphs for an
understanding of the values and index to further accent the present
relationship.

Below is a graph that shows the average housing price over time. The
long run trend of the housing market is upwards with short run
fluctuations. Pictured on this graph in gray are when the U.S was in a
recessionary period as macro-economic conditions can pull down prices.
The biggest drop in the housing market started after the 2008 recession,
which is known as the ``Great Recession''.

\includegraphics{test_files/figure-latex/unnamed-chunk-1-1.pdf}

Below are two types of graphs that show how economic uncertainty and
unemployment fluctuate with prices. While there is some short run
volatility there is no relationship between these two and the long run
trend of increasing housing prices. Overall, these two can be used to
describe the health of the macroeconomy and only when the macroeconomy
is in a recession may we see a short run relationship. However, we see
that during two recessions the prices of the housing market continued to
rise and as such the fall in prices during the great recession was not
due to the macroeconomy but rather the collapse of a force driving up
the prices.

\includegraphics{test_files/figure-latex/unnamed-chunk-2-1.pdf}
\includegraphics{test_files/figure-latex/unnamed-chunk-3-1.pdf}

\includegraphics{test_files/figure-latex/unnamed-chunk-4-1.pdf}

\includegraphics{test_files/figure-latex/unnamed-chunk-5-1.pdf}

\section{The Housing Bubble}\label{the-housing-bubble}

From 1990-2007 the housing market was in a bubble. In a normal market
the price of a good is at the equilibrium found between supply and
demand. People's demand is inversely related to price, i.e if a good is
priced higher than less people will want it. Firms want to price their
goods as high as possible so the price is positively related to price.
The intersection of the supply of good and the demand for them is called
the market equilibrium and the forces of the market push that price to
be at that level. This concept is critically to the state of the housing
market during this time.

There are two graphs below that focus on the supply fluctuations on the
market but this analysis will focus on the period of 1990-2007 which was
before the great recession. The graph below shows how the supply of
houses fluctuates over time. The black line is how many new houses are
created, and the red line is the ratio of the new houses created divided
by the new houses sold. These two values give us a clear insight into
the supply and demand. We see from 1990 to 2000 that new houses keep
being introduced to the market. While the physical supply of houses is
increasing, it is important to note that when a house is bought it is
removed from the market. So, if 1 new house is built a day but 2 houses
are bought that day then the supply is actually falling. The ratio of
the new houses to new houses sold captures this value. We see that
during the period of 1990 to 2000 that the ratio is falling and as such
can conclude that there is no supply issue in this market during this
time. What we can conclude from this is that holding demand constant the
housing prices should only increase slightly at the rate of inflation.

While from 2000-2005 the number of houses entering the market has fallen
the stability in the ratio shows that on the supply side there is not a
disbalance in the market. However, from 2005 onwards we see a lot of
houses enter the market and after that we see a massive spike in the
ratio. This means that there are more new houses in the market than ones
being bought, in other words we should have a housing surplus holding
demand constant. When the supply is greater than demand we expect the
prices of houses to fall.

An indexed graph with a base year of 2000 is included below to further
demonstrate the relationship between the New Homes Created and Ratio of
New Homes purchased.

\includegraphics{test_files/figure-latex/unnamed-chunk-6-1.pdf}
\includegraphics{test_files/figure-latex/unnamed-chunk-6-2.pdf}

\includegraphics{test_files/figure-latex/unnamed-chunk-7-1.pdf}

In our previous analysis we held demand constant and the below graph
shows why this assumption is valid. People buy a home to live in it and
so we can proxy the homeownership rate as a measure of need. We infer a
stable market to proxy need as demand. Some people cannot buy homes
because they do not have the means hence, we cannot expect the rate to
be at 100\%. Instead, as more people grow up and can buy homes if there
is a higher demand for homeownership then the rate will be pushed
upwards. Since the fluctuations in the rate are small we can presume
that throughout this period the demand for people with the means to buy
a home has remained the same.

An indexed graph with a base year of 2000 is included below to further
demonstrate the stability present within the demand side of the market.

\includegraphics{test_files/figure-latex/unnamed-chunk-8-1.pdf}

\includegraphics{test_files/figure-latex/unnamed-chunk-9-1.pdf}

Noted previously when looking at the housing supply before the great
recession we expect to see small increases in the housing prices and
then a fall in prices from 2005-2007. However, looking at the below
graph we do not see this relationship to be true at all. The price was
rising significantly throughout the entire period and rose very fast
after 2005. For emphasis we can see in red that the supply is increasing
and if the demand was held constant then the prices should not be
increasing. This disbalance in the forces of supply and demand is what
it means for a market to be in a bubble. The prices were rising the
entire period in spite of stability in the supply which should have seen
lower prices. The reason that this betrayed our expectations is because
of the nature of a bubble. Homeownership rate changes can capture a new
homeowner meeting their needs by buying a home however this measure
fails to capture a person buying another home. This person has no need
for another home and the only reason they would be it is for a
speculative reason. People looked at the rising housing prices and
thought that houses would continue to grow in value further artificially
boosting the demand. However since this demand is artificial and not
based off of need it results in prices rising to meet expectations until
they inevitably fall and goes down to the true demand level.

\includegraphics{test_files/figure-latex/unnamed-chunk-10-1.pdf}

\includegraphics{test_files/figure-latex/unnamed-chunk-11-1.pdf}

This is a pivotal piece of history in the housing market and outlines
that the movements in the housing market is not inherently related to
business cycles since we now see that this is the reason that prices
fell during the great recession. However, the most important reason to
understand this is that it shows blindly looking at the trends in
housing prices over time without considering other forces will result in
the wrong picture. While housing prices are currently rising we must
look at the underlying forces to see if that trend will continue.

\section{The Current Market}\label{the-current-market}

In recent years the Covid 19 pandemic uprooted everything and the
economy was no exception. Even though it started five years ago it's
impacts are still felt to this day and the current state of the housing
market is a perfect example of this. While business cycles do not impact
the long run prices of the housing market that is not to say it is not
impacted by the macro economy. The market is subject to inflation and
during the Covid 19 pandemic inflation grew immensely. As such part of
the increase in the prices in the housing market can be linked to this
element. However, this report will detail the other driving force behind
the housing market prices.

Below is a graph that in black shows the homeownership rate over time
and in red shows the occupied housing units. We can represent the
homeownership rate as (number of people who own homes) / (total number
of people). A change in the homeownership rate can be modeled by
(existing homeowners + new homeowners) / (total number of people). If we
look at the trend depicted in the graph we see that the number of
occupied homes increases with respect to time and assuming the majority
of those changes comes from first time home owners we would expect the
home ownership rate to increase. However, this is not occurring and that
is because the total population is also increasing. The key element to
analyze is not the volume but instead how the population of people and
houses grows and move together over time.

\includegraphics{test_files/figure-latex/unnamed-chunk-12-1.pdf}

\includegraphics{test_files/figure-latex/unnamed-chunk-13-1.pdf}

Focusing on analyzing the relationship between the homeownership rate
and occupied houses, the two graphs below are snapshots of two different
periods. In the first period of 2017-2020 we see the homeownership rate
is moving with the occupied houses. Meaning that the number of people
buying homes is outpacing the number of people entering the population.
Yet in 2020-2025 we do not see the same picture. The homeownership has a
brief spike due to the pandemic pushing people to buy more homes but the
homeownership rate quickly falls. The number of occupied houses
continues to grow which would suggest that the population growth has
outpaced the growth of the housing market. This suggestion has two vital
implications that involve the context of covid to illuminate. 1. During
the pandemic population growth was very low meaning that a greater
negative impact to the supply of homes had to occur for this
relationship to appear. 2. Covid disrupted supply chains everywhere
stalling the construction of houses. These factors illuminate that the
number of houses in the market has drastically fallen meaning that
unlike the housing bubble there is an actual supply demand disbalance in
the market. The supply of houses is not meeting the demand for houses
which is called a shortage and is an upward pressure on prices. Meaning
that if situations stay constant the housing market will continue to
surge in prices.

For the indexed graphs since the year 2000 is not in the subset of years
considered they instead are indexed around 2017. A version indexed
around 2000 is included but the trend is harder to see. However, in the
set of graphs indexed around 2017 the previously outlined relationship
is clear to see as the values follow eachother pre covid but begin to
diverge post covid.

\includegraphics{test_files/figure-latex/unnamed-chunk-14-1.pdf}
\includegraphics{test_files/figure-latex/unnamed-chunk-14-2.pdf}
\includegraphics{test_files/figure-latex/unnamed-chunk-15-1.pdf}
\includegraphics{test_files/figure-latex/unnamed-chunk-15-2.pdf}

\includegraphics{test_files/figure-latex/unnamed-chunk-16-1.pdf}
\includegraphics{test_files/figure-latex/unnamed-chunk-16-2.pdf}

\section{Should a House be bought:}\label{should-a-house-be-bought}

Purchasing a house during a bubble is unwise however, we are not in
those conditions right now. The market is currently boosted by high
inflation and a shortage. Currently terrible economic policies run
rampant and inflation seems like it will only continue to rise. However,
the fate of the shortage is unknown. It does seem like there are more
houses being introduced to the market and the prices might begin to
fall. It is potentially a risk to buy many houses right now due to
instability and the shortage potentially dropping prices. If one is in
need of a house and wants to buy one right now then it is not an unwise
decision but one could also wait and see if the shortage can be meet and
in term drop prices.

\begin{verbatim}
## i Capturing R dependencies
\end{verbatim}

\begin{verbatim}
## v Found 64 dependencies
\end{verbatim}

\end{document}
